%%%%%%%%%%%%%%%%%%%%%%%%%%%%%%%%%%%%%%%
% Hieu Do - Resume
% 7/26/2016
%
% Reference:
% Debarghya Das (http://debarghyadas.com)

\documentclass[]{hieudo-build}


\begin{document}

%%%%%%%%%%%%%%%%%%%%%%%%%%%%%%%%%%%%%%
%
%     TITLE NAME
%
%%%%%%%%%%%%%%%%%%%%%%%%%%%%%%%%%%%%%%
\namesection{Jean Fauquenot}
{\urlstyle{same}
	\faEnvelope \href{mailto:jean.fauquenot@gmail.com}{ jean.fauquenot@gmail.com}\\
 	%\faHome \href{https://jean.fauquenot.net}{ jean.fauquenot.net}\\
	\faGithub \href{https://github.com/Phibonacci}{ github.com/Phibonacci}\\
    \faPhone { 06.95.98.59.68}
	%\faLinkedinSquare \href{https://www.linkedin.com/in/jean-fauquenot-16480666}{ jean-fauquenot-16480666}
}

%\color{subheadings}\raggedright
\fontspec[Path = fonts/NiveauGrotesk/]{Niveau-Bol}\fontsize{12pt}{12pt}\selectfont\bfseries\raggedright{} \normalfont


%%%%%%%%%%%%%%%%%%%%%%%%%%%%%%%%%%%%%%
%
%     COLUMN ONE
%
%%%%%%%%%%%%%%%%%%%%%%%%%%%%%%%%%%%%%%
\begin{minipage}[t]{0.34\textwidth} 

%%%%%%%%%%%%%%%%%%%%%%%%%%%%%%%%%%%%%%
%     EDUCATION
%%%%%%%%%%%%%%%%%%%%%%%%%%%%%%%%%%%%%%
\section{Formation} 

\subsection{Epitech Paris}
\descript{Master en Informatique}
\sectionsep

\subsection{Griffith College Dublin}
\descript{Bachelor of Science (Hons) in Computing Science}
Année à l'étranger.
\sectionsep

\subsection{Baccalauréat Scientifique}
Lycée Hugh Capet, Senlis
\sectionsep

%%%%%%%%%%%%%%%%%%%%%%%%%%%%%%%%%%%%%%
%     SKILLS
%%%%%%%%%%%%%%%%%%%%%%%%%%%%%%%%%%%%%%
\section{Compétences}
% \subsection{Programming}
\location{Langages favoris :}
C++, C\#, C, Bash, Lua, Python, JavaScript \\ 

\location{Quelques logiciels en vrac :}
UNIX-like, Windows, Git, Hg, Emacs, Visual Studio, LaTeX, Wireshark, zsh \\ 

\location{Langues : }
Français : langue maternelle \\
Anglais : parlé couramment \\

% \sectionsep
% \subsection{Languages}
% \location{Native fluency:} English, Vietnamese\\
\sectionsep

%%%%%%%%%%%%%%%%%%%%%%%%%%%%%%%%%%%%%%
%     CRITERES
%%%%%%%%%%%%%%%%%%%%%%%%%%%%%%%%%%%%%%
\section{Critères}
% \subsection{Programming}
\location{Découvrire des langages :}
Rust, Go, Haskell, F\# ...  \\ 
\location{Résoudre des problèmes}
\location{Un sujet de fond motivant}

% \sectionsep
% \subsection{Languages}
% \location{Native fluency:} English, Vietnamese\\
\sectionsep

% %%%%%%%%%%%%%%%%%%%%%%%%%%%%%%%%%%%%%%
% %     HACKATHONS
% %%%%%%%%%%%%%%%%%%%%%%%%%%%%%%%%%%%%%%
% \section{Hackathons}
% HackMIT \textbullet{} hackNY \\
% WearHacks NY \textbullet{} Hackademics VN \\
% \sectionsep


%%%%%%%%%%%%%%%%%%%%%%%%%%%%%%%%%%%%%%
%     COURSEWORK
%%%%%%%%%%%%%%%%%%%%%%%%%%%%%%%%%%%%%%
% \section{Coursework}
% Data Structure \\
% Algorithms \\
% Discrete Mathematics \\
% Artificial Intelligence \\
% Computer Architecture \\
% Computer Networking \\
% Data Analysis \\
% Object-Oriented Programming\\
% \sectionsep

%%%%%%%%%%%%%%%%%%%%%%%%%%%%%%%%%%%%%%
%     ADDITIONAL INFORMATION
%%%%%%%%%%%%%%%%%%%%%%%%%%%%%%%%%%%%%%
\section{Centres d'intérêt}
Cinéma : Toujours enthousiaste devant un bon film \\
Sport : Vélo, Course, Crossfit \\
Jeux-vidéo : Terraria, Factorio, Subnautica, Rimworld, Overcooked, Factorio, Hitman BM \\
Code : Jeux, bots, game jams

\sectionsep

% %%%%%%%%%%%%%%%%%%%%%%%%%%%%%%%%%%%%%%
% %     AWARDS
% %%%%%%%%%%%%%%%%%%%%%%%%%%%%%%%%%%%%%%

% \section{Awards} 
% Dean's List\\
% NYU PROMISE Scholarship\\
% Shelby C. Davis Scholarship\\
% President’s Circle Scholarship\\
% \sectionsep

\sectionsep
\DTMsetdatestyle{mylastupdate}
\DTMdisplaydate{\the\year}{\the\month}{\the\day}{-1}

%%%%%%%%%%%%%%%%%%%%%%%%%%%%%%%%%%%%%%
%
%     COLUMN TWO
%
%%%%%%%%%%%%%%%%%%%%%%%%%%%%%%%%%%%%%%
\end{minipage} 
\hfill
\begin{minipage}[t]{0.65\textwidth} 

%%%%%%%%%%%%%%%%%%%%%%%%%%%%%%%%%%%%%%
%     EXPERIENCE
%%%%%%%%%%%%%%%%%%%%%%%%%%%%%%%%%%%%%%
\section{Expériences}

\workplace{AnotherBrain}{Janvier 2019 - Août 2019}\\
\position{Deep Learning: Implémentation de Variational Auto Encoders (VAE)}{Paris}
\vspace{0.9em} % Hacky fix for awkward extra vertical space
\begin{tightemize}
\item Développement backend Python, C++, TensorRT
\item Un peu de front-end avec Vue.js, Node.js, Three.js, Redis
\end{tightemize}
\sectionsep

\workplace{STET (via SSII)}{depuis Février 2018}\\
\position{Développement backend de Paylib, paiement par téléphone}{Paris}
\vspace{0.9em} % Hacky fix for awkward extra vertical space
\begin{tightemize}
\item C, C++, PostgreSQL, Bash, réseau, parallélisme, linux
\end{tightemize}
\sectionsep

\workplace{Credit Agricole CIB (via SSII)}{Mai 2017 - February 2018}\\
\position{Migration de produits financiers vers un format unique}{Paris}
\begin{tightemize}
\item C\#, SQL
\end{tightemize}
\sectionsep

\workplace{Epitech GameDevLab}{Septembre 2014 — Mars 2015}\\
\position{Assistant au laboratoire de jeux vidéo d’Epitech}{Paris}
\begin{tightemize}
\item C++, C\# Unity
\end{tightemize}
\sectionsep

% \workplace{Ecole 42}{Juillet 2013 — Décembre 2013} \\
% \position{Installation, maintenance et assistanat pédagogique}{Paris}
% % \vspace{\topsep} % Hacky fix for awkward extra vertical space
% \sectionsep

% \runsubsection{The Westminster News} \\
% \descript{Co-Editors-in-Chief (2014) | Layout Editor (2012) }
% \location{Sep 2012 – May 2015 | Simsbury, CT}
% \begin{tightemize}
% \item Led 30+ staff at a 400-student boarding school to publish the monthly school newspaper.
% \item Consulted in layout design and technology application.
% \end{tightemize}
% \sectionsep

% \runsubsection{VietAbroader Organization} \\
% \descript{Program Assistant Manager | VAPedia Associate }
% \location{Mar 2013 – Mar 2014 | Ho Chi Minh City, Vietnam}
% \begin{tightemize}
% \item VietAbroader is a non-profit organization run by Vietnamese students abroad.
% \item Designed the high school content for the VAPedia website.
% \item Assisted in recruiting guest speakers and volunteers for VietAbroader Study-Abroad Conference 2013.
% \end{tightemize}
% \sectionsep

%%%%%%%%%%%%%%%%%%%%%%%%%%%%%%%%%%%%%%
%     PROJECTS
%%%%%%%%%%%%%%%%%%%%%%%%%%%%%%%%%%%%%%
\section{Projets}

\runsubsection{Gomoku}
\descript{C++, Qt5}
Un projet d’intelligence artificielle pour le jeu de plateau, ”Gomoku”. Implémentation de l’algorithme Minimax.
\sectionsep

\runsubsection{R-Type}
\descript{C++11, SFML2}
%\location{2nd Prize, Hackademics VN 2015}
Un jeu vidéo, de type Shoot’em up, en réseau fonctionnant avec des librairies partagées et des fichiers de configuration en JSON.
\sectionsep

\runsubsection{jChessFX}
\descript{Java, JavaFX}
Un jeu d'échec en Java dans le cadre d'un module de GUI.
\sectionsep

\runsubsection{FTrace}
\descript{C}
Suivant le principe du programme unix strace, ftrace génère un graph représentant tous les appels système d'un logiciel.
\sectionsep

\runsubsection{PROShine}
\descript{C\#, Lua, WPF/XAML, Wireshark}
Un bot open source pour un jeu permettant de créer des scripts Lua pour automatiser ses actions. \\
\href{https://github.com/Silv3rPRO/proshine/}{github.com/Silv3rPRO/proshine} (bot) \\
\href{https://github.com/g0ldPRO}{github.com/g0ldPRO} (scripts)
\sectionsep

\end{minipage} 

\end{document}  
